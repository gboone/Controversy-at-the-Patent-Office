% !TEX encoding = UTF-8 Unicode
%% LaTeX - Article customise
\documentclass[pdftex,12pt,letterpaper]{article}
\usepackage[margin=1in]{geometry}
\usepackage[american]{babel}
\usepackage[style=authoryear]{biblatex}
\usepackage{amsmath,amssymb,amsfonts,csquotes,epigraph,float,hyphenat,footnote,titling,setspace,hyperref}
\bibliography{../final.bib}

%fontspec,xunicode,xltxtra,color,longtable,rotating,hhline,polyglossia,graphicx}

\newcommand{\subtitle}[1]{%
  \posttitle{%
  \par\end{center}
  \begin{center}\large#1\end{center}
  \vskip0.5em}%
}
\doublespacing
%\raggedright
\title{Tussle over Technology: Controversy \& the Patent Act}
\author{Greg Boone \\ Knowledge and Power: Cultures of Expertise \\ David Ribes}
\date{Fall 2012}
\begin{document}
\maketitle
%\tableofcontents
% \autocite{Sorenson:1993:biblatex}.

% \begin{abstract}
% The United States Constitution instructed the First Congress of the United States to enact laws ``securing for limited Times to Authors and Inventors the exclusive Right to their respective Writings and Discoveries.'' This instruction, commonly referred to as the Intellectual Property Clause, demanded a set of laws based on the broad idea that inventors are best incentivized and remunerated for their works through a system of guaranteed exclusive rights. What exactly the scope of those rights were, who had authority to hand them out, and under what conditions they would be given, were all left to Congress to sort out; it was among the earliest controversies over science and technology in the United States of America.
% \end{abstract}

\section*{Introduction}

\epigraph{The Congress shall have the Power...To promote the Progress of Science and useful Arts, by securing for limited Times to Authors and Inventors the exclusive Right to their respective Writings and Discoveries.}{U.S. Constitution}

In the United States of America, the working definition of a patent is best encapsulated not in US Code Title 35 (The Patent Law), but from the United Nations' World Intellectual Property Organization (WIPO). One of a handful of specialized agencies, whose mission is to ``to promote innovation and creativity for the economic, social and cultural development of all countries, through a balanced and effective international intellectual property system.'' Essentially, WIPO seeks to create a predictable and equitable system for intellectual property around the world through the administration of treaties, working with countries to nurture productive intellectual property (IP) systems, and building worldwide ``understanding of and respect for IP'' \autocite{WIPO2012}. Today, the American IP regime is as much shaped by organizations like WIPO and international agreements as by the U.S. Congress. Accordingly, a patent:

\begin{quote}is a document, issued, upon application, by a government office (or a regional office acting for several countries), which describes an invention and creates a legal situation in which the patented invention can normally only be exploited (manufactured, used, sold, imported) with the authorization of the owner of the patent \autocite{WIPO}.
\end{quote}

Like a corporation, on its face a patent is nothing more than a legal document. The power embodied by that document, however, is mighty. A patent grants to an inventor the power to exclude; the power to say no, to determine who is allowed to `practice' or ``make, use, or build'' the disclosed invention. The WIPO Handbook elaborates: ``The effects of the grant of a patent are that the patented invention may not be \emph{exploited in the country} by persons other than the owner of the patent \emph{unless the owner agrees to such exploitation} \autocite[emphasis added]{WIPO},'' in exchange the inventor must disclose in some detail how the technology could be made, used, or built. As this paper will discuss, while this underlying idea is by no means new, the details of how the system was formalized changed in response to changing technological landscapes of the time. A travel through the history of patent law in the United States will show that patent laws respond to a specific technological moment by giving society a way of answering the question: what is technology and what do we do with it?

The United States Constitution empowered the First Congress of the United States to enact laws ``to promote the Progress of Science and useful Arts, by securing for limited Times to Authors and Inventors the exclusive Right to their respective Writings and Discoveries'' \autocite[ Art. 8, \S 1, cl. 8]{USConstitution}. The exact scope of those \textit{rights}, who had authority to administer them, under what conditions and for how long they would be given were all left to Congress to sort out; it was among the earliest controversies over science and technology in the United States of America. Article 8, Section 1, Clause 8, better known as the Intellectual Property Clause, demanded a set of laws based on the broad idea that inventors are best incentivized and remunerated for their works through a system of guaranteed exclusive rights. The resulting law, the Patent Act of 1790, created a kind of paradigm for how the 1790-1836 United States understood the ideas of invention, technology, and science. 

Legal scholars have shown over the last century or so that the idea of a patent system itself was not the controversy. On the contrary, argued Edward C. Walterscheid, the United States as individual colonies almost all had some kind of patent system long before the First Congress passed the Patent Act of 1790. Both Carolinas had patent systems, though they worked differently; Connecticuit had a third kind of operation for its patent system. These different systems were all derived from the patent laws of England, their one time colonial ruler. \autocite{Walterscheid1997} That system was established by the Statute of Monopolies in 1623. Frank D. Prager, writing much earlier than Walterscheid, showed that those laws were in turn inspired by French and even earlier Venetian laws protecting inventions \autocite{Prager1944a}. The idea of a system of limited exclusive rights over new and useful things, then was well entrenched in early American society through its colonial heritage.

Though the question of patents as the preferred means for incentivizing innovation was settled by 1790, it was not the only possible solution to the problem. Thomas Jefferson, the third President and primary author of the Declaration of Independence, famously bemoaned this system of exclusivity despite recognizing its utility \autocite{Jefferson1813}. Other means of remunerating inventors for their works include prize systems, where a wealthy individual sets an amount of money available to the first person to accomplish some task---the modern day X-Prize is a good example, and patronage, where an inventor is paid to work on some specific project at the behest of a wealthy individual or family---Machiavelli, for example, did much of his work under for a patron, Soderini, rather than an exclusive copyright. Some argue the system of exclusion is flawed because ``it is the action of the thinking power called an idea, which an individual may exclusively possess as long as he keeps it to himself; but the moment it is divulged, it forces itself into the possession of every one'' \autocite{Jefferson1813}. For Jefferson, and those like him, ideas belonged to the public once they were released from the private mind. Giving someone a monopoly over ideas, then, was as ridiculous as giving someone control over air. These criticisms would inform the basis of the controversies over the 1790 Patent Act's eventual operation.

Jefferson reluctantly accepted the Intellectual Property Clause and later his duty to enforce it as Secretary of State. He was what legal scholars might call a \emph{desirable society utilitarian}. In his \emph{Theories of Intellectual Property}, Fisher argued for four broad ways of thinking about intellectual property: 1) the ``utilitarian rationale,'' 2) a ``natural law'' perspective, 3) the humanist perspective, and 4) the ``desirable society'' perspective. The first and the last are similar but with different ideas about what these monopolies can accomplish. To Jefferson's (and, he would argue America's) chagrin ``the exclusive right to invention'' is given ``for the benefit of society,'' at the risk of ``embarrassment'' from its unintended consequences \autocite{Jefferson1813}. At the time he wrote these words to Mr. Isaac McPherson in 1813, the United States already had patent law. As Edward C. Walterscheid tells in his 1997 article \emph{Charting a Novel Course: The Creation of the Patent Act of 1790} there was robust debate around what that first Act would look like. Input came from Washington, Jefferson, inventors and many others regard the form and scope of the patent system.

In the end, the Patent Acts of 1790 and 1793 served as a paradigm responding to a specific set of problems unique to the first 40 years of United States industry. What little legislation there was regarding patents in the years intervening it and the Act of 1836 served to articulate the paradigm in the same way that ``normal science'' does for scientific paradigms \autocite{Kuhn1962}. Supreme Court cases, too define the boundaries and limits of the law as it applies in that context. This period of time might be called ``Normal Legislation'', in which the established paradigm runs its course until societal progress and norms force a shift toward a new paradigm.

This paper is less concerned with the virtue of the law, about which a great deal has been written in legal journals that will inform this work. Rather, it will argue that patent acts are a kind of Kuhnian paradigm. The law will settle a set of specific questions and problems and serve as the answer to all disputes until such a time when the relevant parties no longer find the law adequate. The following will show that the Patent Act of 1790 emerged out of a complex debate about what constituted technology in the early years of the American Republic. It will examine the debate and proposals that interpreted the only precedent Congress had in its first three years: the Constitution.

\begin{enumerate}
\item{Pre-paradigmatic Environment: The Colonies, Confederation, and the Constitution}
	\begin{enumerate}
	\item{The English Patent System}
	\item{Colonial Patent Laws}
	\item{Technologies Under The Articles of Confederation}
	\end{enumerate}

\item{Between Paradigms: Controversy in Congress}
	\begin{enumerate}
	\item{State of the Art: What was Invented?}
	\item{What to do About Existing Technologies?}
	\item{Foreign Technologies: Are they patentable?}
	\end{enumerate}
	
\item{A New Technological Order: The Patent Act of 1790}
	\begin{enumerate}
	\item{What is New Technology?}
	\item{According to Whom?}
	\item{What was Included and Excluded?}
	\end{enumerate}
	
\item{Articulation of the Paradigm: 1790-1836}
	\begin{enumerate}
		\item{Epstein: Policy Paradigms}
		\item{Patent System as a Boundary Object?}
	\end{enumerate}

\item{Conclusion: Patent Acts as Paradigm-Defining Works}
\end{enumerate}
% \cite{Dobyns1997} 

\printbibliography
\end{document}