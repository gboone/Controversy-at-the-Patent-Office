% !TEX encoding = UTF-8 Unicode
%% LaTeX - Article customise
\documentclass[pdftex,11pt,letterpaper]{article}
\usepackage[]{geometry}
%margin=1in
\usepackage[american]{babel}
\usepackage[style=authoryear]{biblatex}
\usepackage{amsmath,amssymb,amsfonts,csquotes,epigraph,float,hyphenat,footnote,titling,setspace,hyperref}
\bibliography{../final.bib}

%fontspec,xunicode,xltxtra,color,longtable,rotating,hhline,polyglossia,graphicx}

\newcommand{\subtitle}[1]{%
  \posttitle{%
  \par\end{center}
  \begin{center}\large#1\end{center}
  \vskip0.5em}%
}
\doublespacing
%\raggedright
\title{Tussle over Technology: Controversy \& the Patent Act}
\author{Greg Boone \\ Knowledge and Power: Cultures of Expertise \\ David Ribes}
\date{Fall 2012}
\begin{document}
\maketitle
%\tableofcontents
% \autocite{Sorenson:1993:biblatex}.

% \begin{abstract}
% The United States Constitution instructed the First Congress of the United States to enact laws ``securing for limited Times to Authors and Inventors the exclusive Right to their respective Writings and Discoveries.'' This instruction, commonly referred to as the Intellectual Property Clause, demanded a set of laws based on the broad idea that inventors are best incentivized and remunerated for their works through a system of guaranteed exclusive rights. What exactly the scope of those rights were, who had authority to hand them out, and under what conditions they would be given, were all left to Congress to sort out; it was among the earliest controversies over science and technology in the United States of America.
% \end{abstract}

\section{Introduction}

\epigraph{The Congress shall have the Power\ldots To promote the Progress of Science and useful Arts, by securing for limited Times to Authors and Inventors the exclusive Right to their respective Writings and Discoveries.}{U.S. Constitution}

In the United States of America, the working definition of a patent is best encapsulated not in US Code Title 35 (The Patent Law), but from the United Nations' World Intellectual Property Organization (WIPO). One of a handful of specialized agencies, whose mission is to ``to promote\ldots~a balanced and effective international intellectual property system.'' Essentially, WIPO seeks to create a predictable and equitable system for intellectual property around the world through the administration of treaties, working with countries to nurture productive intellectual property (IP) systems, and building worldwide ``understanding of and respect for IP'' \autocite{WIPO2012}. Today, the American IP regime is as much shaped by organizations like WIPO and international agreements as by the U.S. Congress. Accordingly, a patent:

\begin{quote}is a document, issued, upon application, by a government office (or a regional office acting for several countries), which describes an invention and creates a legal situation in which the patented invention can normally only be exploited (manufactured, used, sold, imported) with the authorization of the owner of the patent \autocite{WIPO}.
\end{quote}

Like a corporation, on its face a patent is nothing more than a legal document. The power embodied by that document, however, is mighty. A patent grants to an inventor the power to exclude; the power to say no, to determine who is allowed to `practice' or ``make, use, or build'' the disclosed invention. The WIPO Handbook elaborates: ``The effects of the grant of a patent are that the patented invention may not be \emph{exploited in the country} by persons other than the owner of the patent \emph{unless the owner agrees to such exploitation}'' \autocite[emphasis added]{WIPO}, in exchange the inventor must disclose in some detail how the technology could be made, used, or built. As this paper will discuss, while this underlying idea is by no means new, the details of how the system was formalized changed in response to changing technological landscapes of the time. A travel through the history of patent law in the United States will show that patent laws respond to a specific technological moment by giving society a way of answering the question: what is technology and what do we do with it?

The United States Constitution empowered the First Congress of the United States to enact laws ``to promote the Progress of Science and useful Arts, by securing for limited Times to Authors and Inventors the exclusive Right to their respective Writings and Discoveries'' \autocite[Art. 1, \S 8, Cl. 8.]{USConstitution}. The exact scope of those \textit{rights}, who had authority to administer them, under what conditions and for how long they would be given were all left to Congress to sort out; it was among the earliest controversies over science and technology in the United States of America. Article 8, Section 1, Clause 8, better known as the Intellectual Property Clause, demanded a set of laws based on the broad idea that inventors are best incentivized and remunerated for their works through a system of guaranteed exclusive rights. The resulting law, the Patent Act of 1790, created a kind of paradigm for how the 1790-1836 United States understood the ideas of invention, technology, and science. 

Though the question of patents as the preferred means for incentivizing innovation was settled by 1790, it was not the only possible solution to the problem. Thomas Jefferson famously bemoaned this system of exclusivity despite recognizing its utility \autocite{Jefferson1813}\footnote{Ironically, Jefferson would become the United States' first patent clerk through his duties as Secretary of State.}. Other means of remunerating inventors for their works include prize systems, where a wealthy individual sets an amount of money available to the first person to accomplish some task, and patronage, where an inventor is paid to work on some specific project at the behest of a wealthy individual or family---Machiavelli, for example, did much of his work under for a patron, Soderini, rather than intellectual property rights. Some argue the system of exclusion is flawed because ``it is the action of the thinking power called an idea, which an individual may exclusively possess as long as he keeps it to himself; but the moment it is divulged, it forces itself into the possession of every one'' \autocite{Jefferson1813}. For Jefferson, and those like him, ideas belonged to the public once they were released from the private mind. Giving someone a monopoly over ideas, then, was as ridiculous as giving someone control over air. These criticisms would inform the basis of the controversies over the 1790 Patent Act's eventual operation.

Jefferson reluctantly accepted the Intellectual Property Clause as what legal scholars might call a \emph{desirable society utilitarian}. In his \emph{Theories of Intellectual Property}, Fisher argued for four broad ways of thinking about intellectual property: 1) the ``utilitarian rationale,'' 2) a ``natural law'' perspective, 3) the humanist perspective, and 4) the ``desirable society'' perspective. The first and the last are similar but with different ideas about what these monopolies can accomplish. To Jefferson's (and, he would argue America's) chagrin ``the exclusive right to invention'' is given ``for the benefit of society,'' at the risk of ``embarrassment'' from its unintended consequences \autocite{Jefferson1813}. At the time he wrote these words to Mr. Isaac McPherson in 1813, the United States already had patent law. As Edward C. Walterscheid tells in his 1997 article \emph{Charting a Novel Course: The Creation of the Patent Act of 1790} there was robust debate around what that first Act would look like. Input came from Washington, Jefferson, inventors and many others regard the form and scope of the patent system.

In the end, the Patent Acts of 1790 and 1793 served as a paradigm responding to a specific set of problems unique to the first 40 years of United States industry. What little legislation there was regarding patents in the years intervening it and the Act of 1836 served to articulate the paradigm in the same way that ``normal science'' does for scientific paradigms \autocite{Kuhn1970}. Supreme Court cases, too define the boundaries and limits of the law as it applies in that context. This period of time might be called ``Normal Legislation'', in which the established paradigm runs its course until societal progress and norms force a shift toward a new paradigm.

This paper is less concerned with the virtue of the law, about which a great deal has been written by legal scholars that will inform this work. Rather, it will examine the debate and proposals that interpreted the only precedent the US Congress had in its first three years---the Constitution---around the question of what constituted technology in the young republic. It holds that the Patent Act of 1790 was paradigmatic in the sense that Hall extrapolated ``policy paradigms'' from Kuhn's scientific and technological paradigms \autocite{Hall1993, Kuhn1970}. The law settled a set of specific questions and problems and served as the answer to all disputes until such a time when the relevant parties no longer found the law adequate.

% section word count: 1197

\section{Intellectual Property Policy Paradigms}

Peter Hall saw policymaking as essentially a process of social learning that ``involves three central variables: the overarching goals that guide policy in a particular field, the techiques or policy instruments used to attain those goals, and the precise settings of these instruments.'' Using British economic policy from 1970-89 as his object of inquiry, he used the social learning mechanism to illustrate a paradigm shift not unlike Thomas Kuhn's scientific paradigms. He argued to think about policy making in three orders of change, the first and second of which he called ``normal policymaking'' \autocite{Hall1993}. The third was ``marked by the radical changes in the overarching terms of policy discourse.'' 

Like a scientific paradigm, first and second order changes do not necessarily lead to third order changes. In Kuhn's terms, normal policymaking has the effect of ``further articulation and specification under new or more stringent conditions'' \autocite{Kuhn1970}. First order change is characterized by ``incrementalism, satisfying, and routinized decision making'' \autocite{Hall1993}. These are policies made frequently. Hall noted first order changes were made every year during the time period he studied.  Second order changes happen less frequently, they ``alter the instruments\ldots without radically altering the hierarchy of goals behind policy.'' In British economic policies, these policies were changes to the controls on lending and monetary policy.

A third order change is much more profound, and far rarer. In British economic policy it was a radical shift from ``Keynesian mode of policymaking to one based on monetarist economic theory'' \autocite{Hall1993}. Third order changes are informed by ``policy experimentation and policy failure'' and can lead to ``movement from one paradigm to another.'' The balance of this paper will argue that the United States Constitution's Intellectual Property Clause prompted the First Congress to establish a paradigm.

%section word count: 290

\section{Pre-paradigmatic Environment: The Colonies Confederation, and the Constitution}

Legal scholars have shown over the last century or so that the idea of a patent system itself was not the controversy. On the contrary, argued Edward C. Walterscheid, intellectual property rights abounded in the American Colonies. Both Carolinas had patent systems, though they worked differently; Connecticut had a third kind. These different systems were all derived from the patent laws of England, their one time colonial ruler \autocite{Walterscheid1997}. That system was established by the Statute of Monopolies in 1623. Frank D. Prager, writing earlier than Walterscheid, showed that those laws were in turn inspired by even earlier Venetian laws protecting inventions and creative works \autocite{Prager1944}. The idea of a system of limited exclusive rights over new and useful things, then was well entrenched in early American society through its colonial heritage. A brief discussion of these systems is necessary to understand the origins of the Intellectual Property Clause and the petitions to Congress and debate on its floor in 1789 and 1790 that led to the creation of the first Patent Act.

Prager (1944) traced the English and European intellectual property systems back to ``a system of privileges developed in Venice,'' far earlier than originally thought. The English Statute of Monopolies and the Statute of Ann were thought to be the first formal intellectual property systems. In 14th Century Venice Prager found a system of guilds sustaining their monopolies over different areas through legally codified privileges. Those privileges turned into exclusive rights when ``a German painter'' living in Venice eventually ``obtained, not only a privilege\ldots~but an actual patent,'' a right to enjoin others from practicing his invention. The Venetian system had all the marks of a formal patent system including requirements that the invention be novel, useful, and inventive---that is, not simply discovery of a scientific truth \autocite{Prager1944}. They were given to inventors for limited times, and were awarded based on expert testimony and ``interviews rather than on a record and specification in writing.'' These remain fundamental properties of patentability today.

By the time the American Colonies were legislating, intellectual property rights were generations old \autocite{Prager1944}. The Continental Congress, the legislative body under the Articles of Confederation, did not issue any patents or enact any laws establishing a patent regime but ``there was nonetheless a patent \emph{custom} extant in both the infant United States and in Great Britain\ldots. That custom, not operating through any formal mechanisms was, by 1787, ``timeworn'' \autocite[emphasis added]{Walterscheid1997}. 
%Patent operation during this period can be perhaps most adequately characterized working through `tacit knowledge' as it has been differentiated from `explicit knowledge' by many STS scholars, notably Harry Collins. Where the former is knowledge which is not written down or ``finds its application in practical settings,'' and the latter is formalized or codified in the form of textbooks or guides \autocite{Collins1985}.

Most colonial governments granted patents through common law customs. The only explicit mention of inventions came from South Carolina's copyright statute which extended the same rights to inventors of ``useful machines'' as authors had over books, maps, and charts. It did not, however, create any kind of formal system by which those rights would be conferred. ``Consequently, the granting of each patent\ldots~required a special act of the legislature'' \autocite{Walterscheid1997, Prager1944}. 

What little policy was made was done during the Colonial and Articles of Confederation eras was first and second order policy change. The Constitution transitioned the United States not only to a federal government, but prompted a third order change to codify a uniform patent system across all the states in the union. After the constitution was adopted, inventors stopped seeking patent in order to see how Congress would act. They sought from the Congress, a policy paradigm over patentability, term length, and formal proceedure.

% section word count: 665

\section{Building a Paradigm: Controversy in Congress}
\epigraph{The productions of genius and the imagination are if possible more really and exclusively property than houses and land and are equally entitled to legal security.}{Noah Webster, (F, Conn.)1788}

The First Congress had it's goal clearly handed to it by the Constitution: devise some kind of system for granting IP rights, but nothing about what those rights should look like. Walterscheid notes that the First congress started receiving petitions from authors inventions waiting for legislation ``almost immediately.'' The authors and inventors wanted answers from Congress about how the law would work, how generic it would be, whether one law would cover both inventors and authors, what would happen to them upon death, and how the assignment process would work. Congress did what it would probably do today on a question it had never before considered, ``it appointed a committee'' \autocite[458]{Walterschied1997}. 

Less than a month after Congress convened inventors David Rumsey John Churchman's petitions were heard in the House posing just these questions. The committee appointed interviewed Churchman about how his navigational instruments worked and committee responded directly to him saying that, with regard to his invention, he should be given some kind of exclusive right for a ``term of years'', but stopped short of saying what that meant or how it would be assigned. In fact, the committee's report generated some controversy when it recommended giving Churchman ``further encouragement to his ingenuity,'' upon proof of his invention's success. The debate was over whether the power given to Congress was to \emph{reward} or simply \emph{encourage} inventors by granting patents \autocite[456-459]{Walterscheid1997}.

Another petition came from inventor Alexander Lewis claiming to have discovered a novel way of ``impelling boats\ldots~through the water, against any current or stream,'' and asking Congress to pass a law securing a twenty-one year term an exclusive right to ``construct boats upon his model.'' Another inventor, Arthur Greer, asked for the same term length, but specified he wanted a patent, for his navigational discoveries, and a third, Englehart Cruse asked for an eight year term over his ``improved steam engine.'' All of these (and presumably more) were ordered to lie on the table \autocite[459]{Walterscheid1997}. Finally, a petition from John Fitch stirred the controversy even more by asking for an exclusive right, not just to build, construct or use the invention, but to enjoin others, like Rumsey, from building improvements upon his steam boat invention. Fitch also claimed to have exclusive rights over his inventions in several states (all of whom, presumably, had different laws on the matter) \autocite[460-462]{Walterscheid1997}. 

The first Patent Act would have to address the concerns of these various petitioners: what should be done with existing patents, foreign inventions (Rumsey's idea originated in France \autocite{Prager1954}), term length, and improvement inventions. These concerns get at a central question of intellectual property, namely, what is a technology, and how long is sufficient enough time to protect it to promote innovation? They were the issues the old and inconsistent systems were failing to address and were crucial for the First Congress in establishing a policy paradigm on the matter.

Noah Webster was the First Congress' largest advocate for intellectual property rights and was anxious to pass legislation enshrining them. He is credited with writing the first proposal for a joint copyright and patent bill. Webster's ideas about how exclusive rights to inventions should be rewarded were philosophically at odds with those of other countries, especially France, where an inventor was required not only to publicly declare his discovery, but bring it in for a public demonstration. Thomas Jefferson, Rumsey's \emph{de facto} patent attorney in France, and Benjamin Franklin, a friend and scientific supporter of Rumsey's, both held influence with Webster \autocite[159-161]{Prager1954}.

If Prager's 1954 account is any indication, the public debate surrounding any patent legislation was a battle between Fitch and Rumsey over a perennial question of patent policy: what shall be done with simultaneous discovery? Rumsey wanted a public examination system like that in France, while Fitch wanted a jury mechanism to determine which invention came first similar to Webster's proposal. On another level Fitch and his supporters also argued that any American system should be predicated on public declarations and disclosures of their inventions, rather than a more secretive but expedient system of closed examination favored by Rumsey. In the end both parties won some part of the argument in the final Patent Act.

Walterscheid's account tells a more complete story, however, when it comes to Congressional action. While Rumsey and Fitch fought over who was the true inventor of the steamboat, the rest of Congress was concerned with the rest of the system. The bill as introduced built an American system based but also improving on the English system. The English too struggled with patentability of improvements, though across the pond that contest took place in the courts \autocite{Walterscheid1997}. In taking up the issues the English system could not resolve and addressing those that came in the form of petitions, the First Congress established a policy paradigm for American intellectual property. 

The bill was ultimately introduced on June 23, 1789, but was ordered to lie on the table because the Congress could not agree on a bill and had other things to consider by the session's end \autocite{Walterscheid1997}.

By the opening of the second session, Congress received more petitions asking for patents for inventions ranging from mills and sundry engines to lightning rods and wheeled boats. The most profound, however, came from President Washington who, in his 1790 address to a joint session, implored the Congress to act on the Intellectual Property Clause and recommended that whatever legislation passed, it include some mechanism for importing useful inventions from abroad. Congress, responding to the president, said it would pass the Act when it was appropriate and the House appointed a committee ``look into'' Washington's recommendations. Congress also had to decide whether the original bill could even be considered in the second session if it were left on the table at the end of the first. Meanwhile Congress was increasingly inundated with petitions, one of which came from Francis Bailey who claimed to have invented a way to mark paper such that it could not be counterfeited. Bailey nearly received what would have been the only individual patent granted through an Act of Congress were it not for the general act pending before the legislature. Clearly, the Congress wanted the granting of exclusive rights to be handled in a single undertaking that established a system, rather than granting patents on a case-by-case basis as in South Carolina's colonial copyright law \autocite[492-498]{walterscheid1997}.

\section{A new Technological Order: The Patent Act of 1790}

Finally, in 1790, a patent bill was introduced that eventually passed and settled all the controversy described \textit{supra}. In it the Senate wanted a law that would establish a system within which instruments of public policy could be created and predictably used. The patent office, for example, was an instrument created during this paradigm to administer and examine patents. These policies and instruments would be used universally in the United States, and any new inventor could access them in a predictable way.

\subsection{What did Congress Say?}

Congress laid out some specifics based on the controversies identified by petitions sent between 1787 and 1790, though mostly these revolved around the formal operation of the system. Congress declared, for example, that inventions worthy of an exclusive right needed to be ``sufficiently useful and important,'' and would only be granted for a period of fourteen (14) years. Examination of the applications would be conducted by a patent board within the State Department consisting of the Secretaries of State and War and the Attorney General. Inventors were required to provide a ``specification in writing and a drawing, and a model if possible'' \autocite[237]{Federico1936}. Indeed Jefferson rejected many an application asking them to resubmit with a model of the invention. Filing it would cost about five dollars.\footnote{``Fifty cents plus ten cents per hundred words of specification, two dollars for making out the patent, one dollar for affixing the Great Seal and twenty cents for endorsement and all other services'' \autocite[237]{Federico1936}.}

One concern of many petitioners regarded whether State patents would be honored as Federal patents. The answer was simply, no. whether those inventors should be granted a patent was be left to the Patent Board, consisting of the Secretary of State, the Attorney General, and the Secretary of War. All those who petitioned Congress from 1787 through 1790 would have to file a patent like all inventors after them would. Rather than attempt to enumerate for the Board a list of criteria, Congress gave broad authority to these three to determine what constituted a ``useful and important'' invention ``not before known or used'' \autocite[Citing the Patent Act of 1790, p. 238]{Federico1936}. 

Federico wrote that Jefferson was likely most influential in this board because of his many useful and important inventions, some of which earned him awards in France, none of which were patented.\footnote{A colorful list of these inventions can be found at \cite[239]{Federico1936}.} These three individuals would ``meet from time to tim and discussed the applications'' pending before the Board. Jefferson, the Secretary of State and prolific man of letters, elucidated the ``rules and regulations '' by which these applications would be evaluated through various letters to citizens looking for guidance on licensing or patenting some invention. As legal scholar P.J. Federico wrote about the Act, ``a personal letter form Jefferson probably represents an action on an application'' \autocite[239-243]{Federico1936}.

The definition of technology quickly became shaped by what it is not. Jefferson was clear in his letter to Isaac McPherson that new uses for old technologies does not merit a patent, an improvement, however, does \autocite{Jefferson1813}. In that same letter, Jefferson identified three rules the Board used that arose largely out of his utilitarian view of intellectual property. Exclusive rights, monopolies as he called them, be granted not to protect a natural right, but one given for the benefit of society. Putting an old technology to new use did not bring enough benefit to merit giving someone a monopoly over that use. Another was that changing the material from which a machine was made did not constitute a patentable technology. ``Making a plowshare of cast rather than of wrought iron'' was not sufficiently inventive because it was still a plowshare. The final rule said that ``a change of form'' was not an invention. A horseshoe was a horseshoe regardless of whether it was ``high-quartered'' or low \autocite{Jefferson1813}. These three rules were not, however, a complete or exhaustive list of the criteria used to judge applications. Federico notes that JEfferson had strong beliefs that ``small devices, obvious improvements or\ldots frivolous devices,'' should not receive the protections of a patent \autocite[241]{Federico1936}. This system was not perfect, but did manage to give both Fitch and Rumsey patents for their respective steamboat inventions.

A report from the clerk's office showed that of the 114 applications the Board received, only 49 patents were granted during the first three years of the Act \autocite[246]{Federico1936}.With regard to the 14 steamboat or engine related patents, each inventor was given a different invention that did not, in the eyes of Congress, conflict with one another. Federico describes the patent regime established as ``conservative'' with a keen attention to the scope of the rights being granted \autocite[]{Federico1936}. This again harkens back to Jefferson's general distain for monopolies except where necessary articulated in his letter to Isaac McPherson. Jefferson wanted to hold `invention' to a high standard. He worried that too many patents for too inconsequential of discoveries would ``embarrass society'' and the young republic \autocite{Jefferson1813}.

\section{Conclusion}

In Hall's account of policy paradigms, the new economic order established was one that was ``based on a fundamentally different conception of how the economy itself worked.'' The shift in British Economics was a radical departure from Keynesian theory and fiscal tools to neoclassical economics and monetary policy tools \autocite{Hall1993}. In the account here, the Americans saw a dramatic shift away from a \emph{custom} to a \emph{codified system} for intellectual property rights. The decay of the former and its failure to address new anomalies in American society combined with a call in the Constitution provoked a third order policy change that set a paradigm around which all future first and second order changes could be conceived.

Just as a Kuhnian scientific paradigm ushers in a new wave of ``normal science,'' \autocite{Kuhn1970}, so policy paradigms bring forth a period of ``normal policymaking.'' Hall says that after a paradigm shift, first and second order policies are adopted to do the work of ``[adjusting] policy without challenging the overall terms'' set by the paradigm \autocite[279]{Hall1993}. Evidence of this in the 1790 Patent Act can be seen immediately following its passage when Congress adopted a resolution establishing a State Department clerk to handle patent applications \autocite{Federico1936}. This is a clear first order change where ``policy at time-1'' is ``deliberately'' and directly influenced by ``the outcomes of policy at time-0.'' They happen frequently and largely respond to unforeseen consequences of past policy and ``new developments'' \autocite[281]{Hall1993}. Other first order activities included the issuing of patents, the collective result of which incrementally defined for the early United States what was and was not technology.

Second order changes are rarer than first order but not as profound as third order, paradigm shifting, actions. According to Hall they are instrumental adjustments that change the mechanisms underlying the current paradigm, but not its ``hierarchy of goals'' \autocite{Hall1993}. Walterscheid notes that nobody was happy with the Patent Act of 1790: not Jefferson, nor the other officials, nor inventors (more than half of whom were rejected by the Patent Board). Congress acted quickly to amend it to essentially remove the examination process that was seen by inventors as stymieing the process. Despite the fact that the system had changed from ``one of registration rather than examination,'' \autocite{Walterscheid1999} this second Act is only a second order change because it preserved the central goal of the 1790 Act which moved the United States from ``a patent custom to a patent system in the United States.'' The standards for obtaining a patent still included non-obviousness and usefulness standards, it was still a formal application and fee based system, and it remained predicated on the idea that the in exchange for a monopoly over their production and use, inventions and discoveries would be disclosed to the public \autocite{Walterscheid1997}.



\singlespacing
\printbibliography
\end{document}